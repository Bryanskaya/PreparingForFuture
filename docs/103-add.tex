\section*{\hfill Приложение Г \hfill}
\addcontentsline{toc}{section}{Приложение Г}
Приведены листинги методов, связанных с USB.

\begin{lstlisting}[label=usb,caption=Основные функции взаимодействия с USB]
namespace Kmbo
{
	class Ukmbo_usb
	{
		FTDI ftdi = new FTDI();
		
		public RObj USB_Open()
		{
			FTDI.FT_STATUS ftStatus;                                    
			
			ftStatus = ftdi.OpenByDescription("USB <-> Serial Cable A");
			
			if (ftStatus == FTDI.FT_STATUS.FT_OK)                      
			{
				ftStatus = ftdi.ResetDevice();                          
				if (ftStatus != FTDI.FT_STATUS.FT_OK) return new RObj((ulong)ftStatus); 
				
				ftStatus = ftdi.SetBitMode(0xff, 0x00);                 
				if (ftStatus != FTDI.FT_STATUS.FT_OK) return new RObj((ulong)ftStatus);   
				
				ftStatus = ftdi.SetTimeouts(1, 1);
				if (ftStatus != FTDI.FT_STATUS.FT_OK) return new RObj((ulong)ftStatus);
				
				return new RObj(0);
			}
			else return new RObj(2);
		}
		
		public RObj USB_Close()
		{
			return new RObj((ulong)ftdi.Close());
		}
		
		public RObj USB_Write(TKmboWrite lpWrite)
		{
			UInt64 res = 0;
			Int16 Reg = 0;
			FTDI.FT_STATUS ftStatus;
			
			UInt16 Adr;
			UInt16 N;
			UInt16[] m = new ushort[62];
			bool s;
			
			Adr = lpWrite.address;
			N = (UInt16)lpWrite.count;
			
			Buffer.BlockCopy(lpWrite.lpBuffer, 0, m, 1, (int)(lpWrite.count * sizeof(UInt16)));
			
			if (lpWrite.synchro == 1) s = true;
			else s = false;
			
			UInt32 RxBytes;                             
			byte[] TxBuffer = new byte[256];            
			byte[] RxBuffer = new byte[2];              
			UInt32 dwBytesToWrite = 0;                  
			UInt32 dwBytesWritten = 0;                  
			UInt32 dwBytesReceived = 0;                
			
			if (N > 62) N = 62;
			m[0] = (UInt16)((Adr << 8) + (N << 1));
			
			UInt16 n = 0;
			UInt16 a;
			
			for (int i = 1; i <= 15; i++)               
			{
				a = (UInt16)(m[0] >> i);
				if ((a & 0x0001) == 0x0001) n++;
			}
			if (n % 2 == 0) m[0]++;
			
			dwBytesToWrite = (UInt32)(N + 1) * 2 + 1;   
			Reg |= 1;                                   
			
			if (s == false) Reg &= ~2;
			else Reg |= 2;
			
			TxBuffer[0] = (byte)Reg;
			for (int i = 1; i <= ((N + 1) * 2); i++)
			if (i % 2 == 0)
			TxBuffer[i] = (byte)(m[(i >> 1) - 1] & 0x00FF);
			else
			TxBuffer[i] = (byte)(m[i >> 1] >> 8);
			
			ftStatus = ftdi.Write(TxBuffer, dwBytesToWrite, ref dwBytesWritten);
			if (ftStatus != FTDI.FT_STATUS.FT_OK) return new RObj((ulong)ftStatus); 
			
			RxBytes = 1;
			ftStatus = ftdi.Read(RxBuffer, RxBytes, ref dwBytesReceived);
			if (ftStatus != FTDI.FT_STATUS.FT_OK) return new RObj((ulong)ftStatus);
			
			if (RxBytes != dwBytesReceived) res = 19;   
			if ((RxBuffer[0] & 0xF9) != 0x20) res = 20; 
			if ((RxBuffer[0] & 0x04) == 0x04) res = 21; 
			if ((RxBuffer[0] & 0x02) == 0x02) res = 22; 
			
			return new RObj(res);
		}
		
		public RObj USB_Read(ref TKmboRead lpRead)
		{
			UInt64 res = 0;
			Int16 Reg = 0;
			FTDI.FT_STATUS ftStatus;
			
			UInt16 Adr;
			UInt16 N;
			UInt16[] m = new UInt16[62];
			bool s;
			
			Adr = lpRead.address;
			N = (UInt16)lpRead.count;
			if (lpRead.synchro == 1) s = true;
			else s = false;
			
			UInt32 RxBytes;                             
			byte[] TxBuffer = new byte[3];              
			byte[] RxBuffer = new byte[129];           
			UInt32 dwBytesToWrite = 0;                 
			UInt32 dwBytesWritten = 0;                  
			UInt32 dwBytesReceived = 0;                 
			
			if (N > 63) N = 0;
			m[0] = (UInt16)((Adr << 8) + (N << 1) + 0x0080);
			
			UInt16 n = 0;
			UInt16 a;
			
			for (int i = 1; i <= 15; i++)               
			{
				a = (UInt16)(m[0] >> i);
				if ((a & 0x0001) == 0x0001) n++;
			}
			if (n % 2 == 0) m[0]++;
			
			if (N == 0) N = 64;
			dwBytesToWrite = 3;                         
			Reg |= 1;                                   
			
			if (s == false) Reg &= ~2;
			else Reg |= 2;
			TxBuffer[0] = (byte)Reg;
			TxBuffer[1] = (byte)(m[0] >> 8);
			TxBuffer[2] = (byte)(m[0] & 0x00FF);
			
			ftStatus = ftdi.Write(TxBuffer, dwBytesToWrite, ref dwBytesWritten);
			if (ftStatus != FTDI.FT_STATUS.FT_OK) return new RObj((ulong)ftStatus);
			
			RxBytes = (UInt32)(2 * N + 1);
			ftStatus = ftdi.Read(RxBuffer, RxBytes, ref dwBytesReceived);
			if (ftStatus != FTDI.FT_STATUS.FT_OK) return new RObj((ulong)ftStatus);
			if (RxBytes != dwBytesReceived) res = 19;   
			
			for (int i = 1; i <= N; i++)                
			m[i] = (UInt16)((RxBuffer[2 * i - 2] << 8) +(RxBuffer[2 * i - 1] & 0x00FF));
			
			Buffer.BlockCopy(m, 1, lpRead.lpBuffer, 0, (int)(lpRead.count * sizeof(UInt16)));
			
			if ((RxBuffer[2 * N] & 0xF9) != 0x20) res = 20; 
			if ((RxBuffer[2 * N] & 0x04) == 0x04) res = 21; 
			if ((RxBuffer[2 * N] & 0x02) == 0x02) res = 22; 
			
			return new RObj(res);
		}
	}
}
\end{lstlisting}