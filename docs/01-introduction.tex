\section*{Введение}
\addcontentsline{toc}{section}{Введение}

\textbf{Головка самонаведения} (ГСН) — автоматическое устройство, которое устанавливается на управляемое средство поражения (ракету, бомбу, торпеду и др.) для обеспечения прямого попадания в объект атаки или сближение на расстояние, меньшее радиуса поражения боевой части средства поражения (СП), то есть для обеспечения высокой точности наведения на цель. ГСН является элементом системы самонаведения. \cite{book1} \\

СП, оборудованное ГСН, может «видеть» «подсвеченную» носителем или ей самой, излучающую или контрастную цель и самостоятельно наводиться на неё, в отличие от ракет, наводимых командным способом. \\

Виды ГСН:
\begin{itemize}
	\item \textbf{РГС (РГС, РЛГСН) — радиолокационная ГСН};
	\item ТГС (ИКГСН) — тепловая, инфракрасная ГСН;
	\item ТВГСН — телевизионная ГС;
	\item ТПВГС - тепловизионная ГС;
	\item ЛГСН - Лазерная ГС.
\end{itemize}

Канал межблочного обмена (КМБО) может быть использован для обмена данными между управляющей ЭВМ и внешними устройствами (в частности, НЧ и СВЧ аппаратурой) в режиме реального времени. Техническим результатом является повышение точности синхронизации обмена данными. \cite{book2} \\

\textbf{Цель} работы - разработать и отладить библиотеку алгоритмов взаимодействия персонального компьютера (ПК) и контрольной измерительной аппаратурой радиолокационной головкой самонаведения (КИА РГС). \\

Выделены следующие \textbf{задачи}:
\begin{enumerate}
	\item[1)] ознакомиться с алгоритмами взаимодействия ПК и КИА РГС;
	\item[2)] изучить протокол канала межблочного обмена (КМБО);
	\item[3)] проработать особенности работы с PCI, USB, Ethernet;
	\item[4)] разработать соответствующую библиотеку алгоритмов и отладить её.
\end{enumerate}