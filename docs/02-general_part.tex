\section*{Основная часть}
\addcontentsline{toc}{section}{Основная часть}
\section{Общие положения}
	\subsection{Канал межблочного обмена (КМБО)}
		\subsubsection{Основные характеристики канала}
		Скорость передачи данных - 1000 Кбит/сек.\\
		
		Расстояние передачи – до 10 м.
		
		\subsubsection{Состав канала}
		В состав канала обмена входят:
		\begin{itemize}
			\item ведущая станция – ПК с контроллером;
			\item оконечные устройства – модули блоков аппаратуры. \\
		\end{itemize}
	
		При необходимости к каналу может подключаться вспомогательное (сервисное) оборудование на правах оконечных устройств.
		Общее количество устройств в составе канала:
		\begin{itemize}
			\item контроллер – 1;
			\item оконечные устройства – до 32. \\
		\end{itemize}
	
		Ведущей станцией, управляющей обменом, является \textbf{контроллер}. \textbf{Оконечные устройства} осуществляют прием информации от контроллера или передачу информации контроллеру под управлением контроллера. \\
		
		Для организации линии связи между контроллером и оконечными устройствами используются две двухпроводные магистральные линии:
		\begin{itemize}
			\item линия передачи тактового сигнала (SYN); 
			\item линия передачи данных (DATA). \\
		\end{itemize}	
	
		Взаимный обмен информацией между контроллером и оконечными устройствами осуществляется \textbf{кадрами информации}.
		
		\subsubsection{Форматы кадров информационного обмена}
		Форматы кадров разделяется на два вида:
		\begin{itemize}
			\item формат кадра передачи информации от контроллера (кадр W);
			\item формат кадра приема информации от оконечного устройства (кадр R). \\
		\end{itemize}
		
		\textbf{Формат кадров} W и R одинаковый и имеет вид: \\
		
		\begin{tabular}{ | c | c | c | c | c| c | }
			\hline
			ЗАГ & ИС & ИС & ... & ИС & БЧ \\
			\hline
		\end{tabular}
		\\
		
		где:
		
		\qquad ЗАГ – заголовок кадра (16 бит);
		
		\qquad ИС – информационное слово (16 бит). Количество информационных слов в кадре – от одного до 64;
		
		\qquad БЧ – бит четности (1 бит). \\
		
		При формировании кадров W на линии данных контроллер является активным на протяжении всего кадра. По окончании кадра контроллер переводится в состояние паузы.\\
		
		При формировании кадров R на линии данных контроллер является активным на протяжении заголовка кадра. По окончании заголовка контроллер переводится в состояние паузы. На протяжении части кадра, соответствующей ИС и БЧ, на линии данных активным является оконечное устройство. \\
		
		По линии тактового сигнала контроллер является активным как во время кадра (W или R), так и в паузе между кадрами. Оконечные устройства по линии тактового сигнала работают только на прием. \\
		
		\textbf{Формат заголовка кадра} имеет вид: \\
		
		\begin{tabular}{ | c | c | c | c | }
			\hline
			АДР & W/R & ЧС & БЗЧ \\
			\hline
		\end{tabular}
		\\
		
		где:
		
		\qquad АДР –  адрес оконечного устройства (8 бит);
		
		\qquad W/R – признак Передача - Чтение (1 бит);
		
		\qquad ЧС   – число информационных слов в кадре (6 бит);
		
		\qquad БЧЗ  – бит четности заголовка (1 бит). \\
		
		\textbf{Адрес оконечного устройства} – положительное двоичное число от 1 до 255. Передача адреса начинается со старшего разряда и кончается младшим разрядом. \\
		
		\textbf{Признак W/R} – 1 бит, имеет значение 0 – передача данных от контроллера оконечному устройству (кадр W), 1 – прием данных от оконечного устройства (кадр R). \\
		
		\textbf{Число информационных слов} в кадре – положительное двоичное число от 1 до 64. Передача ЧС начинается со старшего разряда и кончается младшим разрядом. Код 000000 соответствует числу слов 64. \\
		
		\textbf{Бит четности заголовка} дополняет заголовок до нечетности. \cite{kmbo}\\
		
		Числовая информация передается дополнительным кодом.\\
		
		\subsubsection{Требования к контроллеру}
		Контроллер, как правило, является устройством, управляемым программно процессором ПК. 
		Может находиться в двух состояниях:
		\begin{itemize}
		\item состояние паузы;
		\item состояние работы.
		\end{itemize}
		
		В \textbf{состоянии паузы} контроллер находится в перерыве между кадрами передачи или приема информации. \\
		
		В \textbf{состоянии работы} контроллер осуществляет прием или передачу заданного числа слов. \\
		
		\textbf{При передаче данных (кадр W)} контроллер передает заголовок, заданное число слов и бит четности БЧ. Бит четности вычисляется контроллером аппаратно. После окончания передачи кадра контроллер переходит в состояние паузы. \\
		
		\textbf{При приеме данных (кадр R)} контроллер передает заголовок, после чего принимает заданное число слов и бит четности БЧ. Проверка соответствия бита четности проводится контроллером аппаратно. После окончания приема кадра контроллер переходит в состояние паузы. \\
		
		\textbf{При передаче кадра W}, а также при передаче заголовка кадра R контроллер одновременно производит чтение с линии данных, которые он сам передал, и аппаратно производит сравнение переданных и принятых данных. Результат (1 бит,  0 - при совпадении, 1 - при несовпадении) должен быть доступен для чтения процессору по окончании кадра.
		
		\subsubsection{Требования к оконечному устройству}
		Оконечное устройство, как правило, является аппаратным устройством и функционирует в соответствии с заложенным в него алгоритмом. При этом оконечное устройство может находиться в 4-х состояниях:
		\begin{itemize}
			\item  исходное состояние;
			\item  состояние приема заголовка кадра;
			\item  состояние ожидания паузы;
			\item  состояние работы. \\
		\end{itemize}
	
		После приема заголовка кадра (16 бит) устройство сравнивает адрес АДР в заголовке с собственным адресом.\\
		
		При совпадении адреса и правильном бите четности заголовка БЧЗ оконечное устройство переходит в состояние работы. При этом в зависимости от бита W/R осуществляется прием данных с линии или выдача данных на линию. Устройство принимает или передает число слов, на которое оно настроено аппаратно. По окончании приема или передачи указанного числа слов и бита четности БЧ устройство переходит в состояние ожидания паузы. \\
		
		Характер исполнения принятой устройством информации при несовпадении числа слов в кадре и аппаратно заложенного в устройство числа слов, а также при неправильном бите четности кадра БЧ настоящим протоколом не определяется и устанавливается в каждом случае отдельно. \\
		
		При несовпадении адреса или неправильном бите четности заголовка БЧЗ оконечное устройство переходит в состояние ожидания паузы.
		
	\subsection{USB-КМБО}
	\subsubsection{Общее}
	Контроллер реализован в виде внешнего USB модуля с размерами 30мм $\times$ 85мм $\times$ 145мм и предназначен для сопряжения персонального компьютера (ПК) через порт USB с каналом межблочного обмена (КМБО) и каналами разовых (битовых) команд (РК) и используется в составе автоматизированной контрольно-испытательной аппаратуры (АКИА). \\
	
	В отличие от контроллера PCI-КМБО контроллер USB-КМБО имеет ряд дополнительных функций, позволяющих производить проверку работы аппаратуры при предельных отклонениях временных параметров сигналов КМБО. \\
	
	Контроллер включает в себя ПЛИС (программируемая логическая интегральная схема), реализующую протокол обмена КМБО, приемо-передатчики сигналов КМБО и интерфейсное устройство шины USB (микросхема FT2232H). \cite{FT2232H} \\
	
	FTDI – драйвер и библиотека функций микросхемы FT2232H, используемой в контроллере USB-КМБО. \cite{ftdi}
	
	\subsubsection{Состав изделия}
	В состав изделия входят следующие функциональные узлы.
	\begin{enumerate}
		\item \textbf{ППЗУ} - перепрограммируемое постоянное запоминающее устройство (EEPROM), предназначено для хранения данных конфигурации изделия.
		\item \textbf{Интерфейс ППЗУ} обеспечивает чтение ППЗУ и его программирование по шине USB.
		\item \textbf{Интерфейс USB} обеспечивает протокол обмена изделия с шиной USB.
		\item \textbf{Кварцевый генератор с умножителем} обеспечивает формирование тактовых импульсов с частотами 6, 12 и 48 МГц.
		\item \textbf{Приемный буфер RX} предназначен для приема информации, поступающей с шины USB, с последующей передачей ее в узлы микросхемы EPM9320LC84.
		\item \textbf{Передающий буфер TX} предназначен для передачи информации в шину USB, поступающей от узлов микросхемы EPM9320LC84.
		\item \textbf{Регистр выходных РК} обеспечивает прием и хранение разовых команд РК, поступающих с шины USB в приемный буфер RX.
		\item \textbf{Регистр командного слова} обеспечивает прием и дешифрацию командного слова, поступающего с шины USB в приемный буфер RX.
		\item \textbf{Интерфейс КМБО} обеспечивает формирование W-кадров и заголовков R-кадров информации КМБО, значения которых считываются из приемного буфера RX, а также прием данных R-кадров и запись их в передающий буфер TX. Устройство осуществляет также формирование бита четности в кадрах W, контроль бита четности в кадрах W и R (ВН) и т.д.
		\item \textbf{Регистр входных РК} обеспечивает передачу входных разовых команд РК в передающий буфер TX.
		\item \textbf{Твердотельные реле} предназначены для выдачи выходных РК.
		\item \textbf{Приемопередатчики RS-485} обеспечивают формирование уровней сигналов КМБО.
	\end{enumerate}

	\subsection{PCI-КМБО}
	\subsubsection{Основные части}
	Изделие состоит из следующих функциональных устройств.
	\begin{enumerate}
		\item \textbf{Интерфейс PCI} обеспечивает связь изделия с шиной PCI ПЭВМ. Обмен информацией между изделием и ПЭВМ осуществляется через четыре порта ввода/вывода и один канал прерывания.
		\item \textbf{Конфигурационная память} выполняет координационные функции по распределению ресурсов между устройствами при помощи специальной конфигурационной программы операционной системы ПЭВМ. После загрузки операционной системы в соответствующих регистрах конфигурационной памяти можно прочитать выделенные изделию базовый адрес ввода/вывода и номер канала прерывания.
		\item \textbf{Буферное запоминающее устройство (БЗУ)} обеспечивает хранение передаваемых и принимаемых данных КМБО. БЗУ двухпортовое, через первый порт происходит обмен информацией между БЗУ и ПЭВМ, через второй порт – между БЗУ и КМБО. Обмен между БЗУ и КМБО происходит во время передачи или приема кадров информации КМБО, а между БЗУ и ПЭВМ - в паузах между кадрами.
		\item\textbf{ Счетчик адреса} обеспечивает последовательные запись или чтение информации в/из БЗУ. 
		\item \textbf{Регистры}, предназначенные для управления изделием со стороны ПЭВМ.
		\item \textbf{Устройство синхронизации} обеспечивает синхронную работу составных узлов изделия и формирование структуры кадра.
		\item \textbf{Устройство формирования сигналов КМБО} обеспечивает формирование 
		W-кадров и заголовков R-кадров информации КМБО, значения которых хранятся в БЗУ, а также прием данных R-кадров и запись их в БЗУ. Устройство осуществляет также формирование бита четности в кадрах W, контроль бита четности в кадрах W и R (ВН) и контроль встречной работы (ВР).
		\item \textbf{Формирователь прерываний} обеспечивает формирование сигнала прерываний от таймера и по окончании передачи/приема кадров КМБО. 
		\item \textbf{Таймер} обеспечивает формирование импульсов с программируемым периодом следования. 
		\item \textbf{Счетчик времени}
	\end{enumerate}

	\subsubsection{Режимы изделия}
	Контроллер может находится в следующих режимах.
	\begin{enumerate}
		\item \textbf{Режим ВЫКЛ}\\
		В этом режиме через конфигурационные регистры производится проверка наличия подключенного изделия, чтение базового адреса и номера прерывания, назначенные операционной системой ПЭВМ (персональной электронной вычислительной машиной), и перевод изделия в режим ПОКОЙ.
		
		\item \textbf{Режим ПОКОЙ} \\
		Отличается от режима ВЫКЛ тем, что появляется возможность записи/чтения рабочих регистров изделия для установки режимов его работы. 
		
		\item \textbf{Режим КОНТРОЛЛЕР} \\
		Включается программно. В этом режиме изделие функционирует в режиме ведущей станции (контроллера) и обеспечивает следующие функции:
		\begin{itemize}
			\item обмен информацией по линии связи, скорость задается программно;
			\item хранение передаваемых и принимаемых данных в БЗУ (буферное запоминающее устройство) емкостью в 64 шестнадцатиразрядных слов;
			\item формирование кадров передачи данных (кадры W) и приема данных (кадры R) длиной от 1 до 64 информационных слов;
			\item аппаратное формирование бита четности в кадрах W и аппаратный контроль бита четности в кадрах W и R;
			\item аппаратное побитное сравнение передаваемых и принимаемых данных;
			\item формирование сигнала прерывания в конце каждого кадра.
		\end{itemize}
	\end{enumerate}

	\subsection{Ethernet-КМБО}
	\subsubsection{Общее}
	МС – устройство, сопрягающее персональный компьютер ПК, активный канал и канал радиокоррекции (РК). \\
	
	Применяется в составе аппаратуры контроля головки. Назначение модуля – преобразование интерфейсов КМБО и МКИО, контроль источников напряжения, выдача битовых команд. МС соединятся с ПК по Ethernet интерфейсу, с активным каналом по МКИО, с каналом РК по каналу межблочного обмена (КМБО). \cite{mc}
	
	\subsection{Вывод}
	Необходимо учесть все особенности работы с платами PCI-КМБО, USB-КМБО, Ethernet-КМБО для того, чтобы разработать необходимую библиотеку для информационного взаимодействия ПК и измерительной аппаратуры.
	
	
		
		
		
		
		

		